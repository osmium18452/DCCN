\documentclass{article}
\usepackage[utf8]{inputenc}

\title{Spectral-Spatial Hyper-spectral Image Classification Using Dual-Channel Capsule Networks}
\author{liu wenbo}
\date{March 2020}

\begin{document}

	\maketitle


	\section{ABSTRACT}\label{sec:abstract}
	Deep learning methods have shown their marvel performance on hyper-spectral image (HSI) classification tasks.
	In particular,algorithms based on Convolution Neural Network (CNN) outperforms most of the conventional machine-
	learning based algorithms.
	Recently, a newly proposed neural network called Capsule Network (CapsNet) showed its potential to replace the CNNs
	in various classification tasks with its amazing performance.
	In this paper, we proposed a new CapsNet architecture for HSI classification tasks, called Dual-channel Capsule
	Network (DCCN).
	Our DCCN model extracts features from spectral and special domain respectively with two separate convolution
	channels and then concatenates and feds the features into the capsule layer to classify each of the HSI pixels.
	The model is trained on \textbf{2/3/4/5?} real HSI dataset \textbf{Pavia University, Pavia City, Salinas Valley,
	Salinas A and KSC} and achieved high accuracy.
	We also compared our network with some of the state-of-art models and found our model was superior to those models.


	\section{INTRODUCTION}\label{sec:introduction}


	\section{RELATED WORKS}\label{sec:related-works}

	\subsection{Capsule Network}\label{subsec:capsule-network}

	\subsection{Margin Loss}\label{subsec:margin-loss}


	\section{PROPOSED METHOD}\label{sec:proposed-method}

	\subsection{Network Architecture}\label{subsec:network-architecture}

	\subsection{Data Pre-Process}\label{subsec:data-pre-process}

	\subsection{Training the Network}\label{subsec:training-the-network}


	\section{EXPERIMENT}\label{sec:experiment}

	\subsection{Dataset Introduction}\label{subsec:dataset-introduction}

	\subsection{Environment}\label{subsec:environment}

	\subsection{Result and Analysis}\label{subsec:result-and-analysis}


	\section{DISCUSSION}\label{sec:discussion}

	\subsection{Convergence analysis}\label{subsec:convergence}



	\section{CONCLUSION}\label{sec:conclusion}

\end{document}
