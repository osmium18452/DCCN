\documentclass{article}
\usepackage[utf8]{inputenc}
\usepackage{fullpage}

\title{Spectral-Spatial Hyper-spectral Image Classification Using Dual-Channel Capsule Networks}
\author{liu wenbo}
\date{March 2020}

\begin{document}

	\maketitle


	\section{ABSTRACT}\label{sec:abstract}
	Deep learning methods have shown their marvel performance on hyper-spectral image (HSI) classification tasks.
	In particular,algorithms based on Convolution Neural Network (CNN) outperforms most of the conventional machine-
	learning based algorithms.
	Recently, a newly proposed neural network called Capsule Network (CapsNet) showed its potential to replace the CNNs
	in various classification tasks with its amazing performance.
	In this paper, we proposed a new network architecture based on CapsNet for HSI classification tasks, called
	Dual-channel Capsule Network (DCCN).
	Our DCCN model extracts features from spectral and special domain respectively with two separate convolution
	channels and then concatenates and feds the features into the capsule layer to classify each of the HSI pixels.
	The model is trained and tested on \textbf{(2/3/4/5?)} real HSI dataset \textbf{(Pavia University, Pavia City,
	Salinas Valley, Salinas A and KSC)} and achieved high accuracy.
	We also compared our network with some of the state-of-art models and found our model outperformed these models.
	Among these state-of-art models we also implemented the Capsule Network that has the same architecture as our
	network except the spectral convolution channel, and we found the DCCN model was superior to it as well.

	\noindent \textbf{Key Words:} Dual-Channel Capsule Network, Hyper-Spectral Image, Image Classification.


	\section{INTRODUCTION}\label{sec:introduction}

	The remainder of the paper is organized as follows.
	In section~\ref{sec:related-works}, we introduce some related work of our paper.
	In section~\ref{sec:proposed-method}, we describe the structure of our proposed network, how we pre-process the data
	and how to train the network.
	In section~\ref{sec:experiment}, we present and analyze the result of out experiment.
	In section~\ref{sec:discussion}, we further explore the potential of DCCN\@.
	Finally, in section~\ref{sec:conclusion}, we  conclude this paper with some remarks and hints at plausible future
	research lines.


	\section{RELATED WORK}\label{sec:related-works}

	\subsection{Capsule Network}\label{subsec:capsule-network}

	\subsection{Margin Loss}\label{subsec:margin-loss}


	\section{PROPOSED METHOD}\label{sec:proposed-method}

	\subsection{Network Architecture}\label{subsec:network-architecture}

	\subsection{Data Pre-Process}\label{subsec:data-pre-process}

	\subsection{Train the Network}\label{subsec:train-the-network}


	\section{EXPERIMENT}\label{sec:experiment}

	\subsection{Dataset Introduction}\label{subsec:dataset-introduction}

	\subsection{Environment}\label{subsec:environment}

	\subsection{Result and Analysis}\label{subsec:result-and-analysis}


	\section{DISCUSSION}\label{sec:discussion}

	\subsection{Convergence analysis}\label{subsec:convergence}



	\section{CONCLUSION}\label{sec:conclusion}

\end{document}
